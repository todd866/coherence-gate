\documentclass[11pt,a4paper]{article}
\usepackage[utf8]{inputenc}
\usepackage{amsmath,amssymb}
\usepackage{graphicx}
\usepackage[margin=1in]{geometry}
\usepackage{hyperref}
\usepackage{natbib}
\usepackage{booktabs}

\newcommand{\Hreg}{H_{\mathrm{reg}}}
\newcommand{\Hdyn}{H_{\mathrm{dyn}}}

\title{Thermodynamic Scaling Limits of Algorithmic Oscillatory Synchronization: \\
Why Physical Coupling is Required for Large-Scale Oscillatory Neural Networks}

\author{Ian Todd\\
Sydney Medical School, University of Sydney\\
\texttt{itod2305@uni.sydney.edu.au}}

\date{\today}

\begin{document}

\maketitle

\begin{abstract}
Recent advances in oscillatory graph neural networks demonstrate that phase synchronization provides a powerful mechanism for high-order feature extraction and relational reasoning. However, we show that implementing continuous coupling dynamics on discrete digital processors incurs a fundamental computational burden: $O(N^2)$ operations per timestep for dense coupling or $O(N \cdot d)$ for sparse graphs with average degree $d$. This algorithmic overhead renders large-scale oscillatory networks thermodynamically intractable on von Neumann architectures. By contrast, physical instantiation of coupled oscillators---where coupling emerges from substrate physics (Kirchhoff's laws, optical interference, molecular diffusion) rather than algorithmic computation---eliminates the per-timestep coupling cost entirely. We present numerical evidence that physical implementations achieve $10^2$--$10^3\times$ power reduction compared to GPU simulation, with the advantage scaling as $O(N^2/K)$ where $K$ is the number of coherence monitoring taps. We propose the \textbf{Coherence Gate} as a hardware primitive for energy-efficient oscillatory computing, wherein coherence-threshold gating replaces clock-driven sampling. These results establish fundamental scaling limits for software-based oscillatory neural networks and motivate the development of dedicated neuromorphic oscillator hardware.
\end{abstract}

\section{Introduction}

Oscillatory synchronization has emerged as a compelling computational paradigm for neural networks. The recent HoloGraph architecture \citep{dan2025holograph} demonstrated that Kuramoto-style phase coupling can capture high-order structural relationships in graphs, achieving state-of-the-art performance on molecular property prediction, node classification, and link prediction tasks. This success validates decades of theoretical work on oscillator-based computing \citep{kuramoto1984chemical, acebrón2005kuramoto, hoppensteadt2000pattern}.

However, a critical question remains unaddressed: \emph{can oscillatory neural networks scale?}

Current implementations, including HoloGraph, execute oscillator dynamics on conventional digital hardware (GPUs/TPUs). At each integration timestep, these systems must:
\begin{enumerate}
\item Compute all pairwise phase differences $\phi_j - \phi_i$ for coupled oscillators
\item Evaluate coupling functions (typically $\sin(\phi_j - \phi_i)$)
\item Accumulate forces and update phase states
\item Store intermediate results in memory
\end{enumerate}

For a network of $N$ oscillators with dense coupling, this requires $O(N^2)$ floating-point operations per timestep. Even for sparse graphs with average degree $d$, the cost remains $O(N \cdot d)$ per timestep over potentially millions of integration steps.

This paper establishes that this computational overhead is not merely an implementation detail but a \textbf{fundamental thermodynamic barrier} to scaling oscillatory neural networks on digital hardware. This observation sits within the broader transition toward neuromorphic hardware outlined in recent roadmaps \citep{christensen2022roadmap}; however, we argue that the energy advantage of neuromorphic systems is lost if the coupling term itself remains algorithmic. We then demonstrate that physical instantiation of oscillator coupling---where phase relationships evolve according to substrate physics without algorithmic intervention---eliminates this barrier entirely.

\subsection{The Core Insight}

The key observation is simple but consequential:

\begin{quote}
\emph{Digital simulation computes coupling. Physical hardware embodies coupling.}
\end{quote}

When oscillators are physically coupled (via resistors, capacitors, optical waveguides, or molecular diffusion), their phase relationships evolve according to Kirchhoff's laws or equivalent physical principles. No processor computes $\sin(\phi_j - \phi_i)$; the coupling simply \emph{happens} as a consequence of the substrate's physics.

This distinction has profound implications for power scaling. Digital simulation dissipates energy proportional to the number of operations: $P_{\mathrm{digital}} \propto N^2 \cdot f_{\mathrm{step}} \cdot E_{\mathrm{op}}$, where $f_{\mathrm{step}}$ is the integration frequency and $E_{\mathrm{op}}$ is energy per operation. Physical coupling dissipates energy proportional to the substrate's intrinsic dissipation, independent of network size for fixed oscillator design.

\section{Computational Complexity Analysis}

\subsection{Digital Simulation Cost Model}

Consider a Kuramoto oscillator network with $N$ nodes and coupling adjacency matrix $A$:
\begin{equation}
\frac{d\phi_i}{dt} = \omega_i + \sum_{j=1}^{N} A_{ij} K_{ij} \sin(\phi_j - \phi_i + \alpha_{ij})
\end{equation}

Numerical integration via explicit Euler or Runge-Kutta methods requires, per timestep:
\begin{itemize}
\item \textbf{Phase differences:} $M$ subtractions, where $M = \sum_{i,j} A_{ij}$ is the number of edges
\item \textbf{Coupling evaluation:} $M$ sine evaluations (or lookup table accesses)
\item \textbf{Force accumulation:} $M$ multiply-accumulate operations
\item \textbf{State update:} $N$ additions
\end{itemize}

Total operations per timestep: $\Theta(M) = \Theta(N \cdot d)$ for sparse graphs, $\Theta(N^2)$ for dense graphs.

For a typical simulation window $T$ with timestep $\Delta t$, total operations scale as:
\begin{equation}
\mathrm{Ops}_{\mathrm{digital}} = \frac{T}{\Delta t} \cdot \Theta(N^2)
\end{equation}

Using standard GPU energy estimates of $E_{\mathrm{op}} \approx 10$ pJ/FLOP \citep{jouppi2017datacenter, horowitz20141}:
\begin{equation}
E_{\mathrm{digital}} = \mathrm{Ops}_{\mathrm{digital}} \cdot E_{\mathrm{op}} = \frac{T}{\Delta t} \cdot \Theta(N^2) \cdot 10^{-11} \text{ J}
\end{equation}

\subsection{Physical Coupling Cost Model}

In a physically coupled oscillator array, phase dynamics emerge from substrate physics without algorithmic intervention. The computational cost reduces to:
\begin{enumerate}
\item \textbf{Static power:} Maintaining $N$ oscillators in active state
\item \textbf{Coherence monitoring:} Measuring aggregate synchronization from $K$ sparse taps
\item \textbf{Registration:} Writing output codes when coherence threshold is crossed
\end{enumerate}

Crucially, the coupling itself is ``free''---it occurs through physical component connectivity (resistive, capacitive, inductive, optical, or molecular) without requiring computation.

For coherence monitoring via $K$ sparse taps ($K \ll N$), the operational cost per timestep is:
\begin{equation}
\mathrm{Ops}_{\mathrm{physical}} = \Theta(K)
\end{equation}

The key insight: \textbf{$K$ is independent of $N$}. A network of $10^6$ oscillators requires no more coherence-monitoring operations than a network of $10^2$ oscillators, provided both use the same number of taps.

\subsection{Scaling Advantage}

The energy ratio between digital and physical implementations scales as:
\begin{equation}
\frac{E_{\mathrm{digital}}}{E_{\mathrm{physical}}} = \Theta\left(\frac{N^2}{K}\right)
\end{equation}

For $N = 10^3$ oscillators with $K = 20$ taps, this predicts a $5 \times 10^4$ advantage. For $N = 10^6$, the advantage exceeds $10^{10}$.

\section{Numerical Validation}

To validate the scaling analysis, we implemented both digital simulation and a model of physical coherence-gate operation for a graph coloring constraint satisfaction problem.

\subsection{Experimental Setup}

\textbf{Problem:} Graph coloring on sparse random graphs with average degree $d = 10$.

\textbf{Digital simulation:} Kuramoto dynamics with repulsive coupling ($K = -2.0$) integrated via explicit Euler method with timestep $\Delta t = 1$ ms. Energy cost: 10 pJ per FLOP, counting 4 operations per edge (subtraction, sine, multiplication, accumulation) plus $N$ state updates.

\textbf{Physical model:} Same underlying dynamics, but energy cost limited to coherence monitoring from $K = 20$ sparse taps (measuring Kuramoto order parameter $r = |N^{-1}\sum_i e^{i\phi_i}|$) plus threshold comparison. Coupling cost: zero (physically instantiated).

\textbf{Commit criterion:} Both systems solve the problem when $r > 0.85$.

\subsection{Results}

Figure~\ref{fig:energy} shows cumulative energy dissipation over a 2-second solve window for $N = 100$ oscillators. Digital simulation dissipates $4.30 \times 10^{-5}$ J while the physical model dissipates $5.00 \times 10^{-7}$ J---an \textbf{86$\times$ advantage}.

\begin{figure}[h]
\centering
\includegraphics[width=0.95\textwidth]{figures/fig_energy_comparison.pdf}
\caption{Cumulative energy dissipation for $N=100$ oscillator graph coloring. \textbf{Top:} Digital simulation (red) requires $O(N^2)$ coupling computations per timestep; physical coherence gate (green) requires only $O(K)$ coherence measurements. \textbf{Bottom:} Kuramoto order parameter showing coherence threshold crossing at $t = 0.31$ s.}
\label{fig:energy}
\end{figure}

Figure~\ref{fig:scaling} demonstrates that the advantage scales with system size as predicted by the $O(N^2/K)$ analysis:

\begin{table}[h]
\centering
\begin{tabular}{r|r|r|r}
\toprule
$N$ & Digital (J) & Physical (J) & Advantage \\
\midrule
25 & $1.07 \times 10^{-5}$ & $4.40 \times 10^{-7}$ & 24$\times$ \\
50 & $2.31 \times 10^{-5}$ & $4.60 \times 10^{-7}$ & 50$\times$ \\
100 & $4.30 \times 10^{-5}$ & $5.00 \times 10^{-7}$ & 86$\times$ \\
200 & $8.22 \times 10^{-5}$ & $5.80 \times 10^{-7}$ & 142$\times$ \\
400 & $1.66 \times 10^{-4}$ & $7.40 \times 10^{-7}$ & 224$\times$ \\
\bottomrule
\end{tabular}
\caption{Energy comparison scaling with system size. Advantage grows approximately as $N^2/K$ due to elimination of algorithmic coupling computation.}
\label{tab:scaling}
\end{table}

\begin{figure}[h]
\centering
\includegraphics[width=0.8\textwidth]{figures/fig_scaling.pdf}
\caption{Energy advantage scaling with system size. Measured advantage (black) grows with $N$ as predicted by $O(N^2/K)$ theory (dashed red). Physical coupling eliminates the quadratic computational burden of digital simulation.}
\label{fig:scaling}
\end{figure}

\section{The Coherence Gate Primitive}

Based on these scaling results, we propose the \textbf{Coherence Gate} as a fundamental primitive for energy-efficient oscillatory computing. Analogous to how a transistor gates current based on voltage, a Coherence Gate gates information based on synchronization. The Coherence Gate is defined by:

\begin{enumerate}
\item \textbf{Physical coupling:} Oscillator interactions implemented via substrate physics (Kirchhoff's laws, optical interference, etc.) rather than algorithmic computation
\item \textbf{Sparse coherence monitoring:} Aggregate synchronization measured from $K \ll N$ taps, computing Kuramoto order parameter or equivalent metric
\item \textbf{Threshold-triggered registration:} Discrete output codes emitted only when coherence crosses threshold, replacing clock-driven sampling
\item \textbf{Registration-rate budgeting:} Hardware-enforced limit on output rate $\Hreg \ll \Hdyn$, where $\Hdyn$ is intrinsic dynamics rate
\end{enumerate}

The Coherence Gate performs controlled dimensionality collapse: high-dimensional analog dynamics ($N$ oscillator phases) are reduced to low-dimensional discrete outputs (coherent states distinguishable at $K$ taps) through the system's own physics.

\subsection{Dimensional Bottleneck and Code Formation}

A key property of the Coherence Gate architecture is \textbf{spontaneous discrete symbol formation}. When $K \ll N$, the sparse monitoring path creates a dimensional bottleneck. In the presence of noise, this bottleneck causes the continuous oscillator dynamics to relax onto a finite set of metastable attractor configurations that are distinguishable at the $K$-dimensional readout.

The emitted ``code'' is therefore not an arbitrary snapshot but a discrete, noise-robust symbol corresponding to a stable attractor. This is a thermodynamically favoured consequence of the hardware architecture rather than a separately engineered feature.

\section{Discussion}

\subsection{Implications for HoloGraph and Oscillatory GNNs}

The HoloGraph architecture \citep{dan2025holograph} demonstrated impressive performance on graph learning tasks. However, our analysis reveals that scaling HoloGraph to large graphs ($N > 10^4$) on GPU hardware will encounter fundamental thermodynamic limits. The $O(N^2)$ coupling computation per timestep cannot be eliminated through algorithmic optimization---it is intrinsic to the digital simulation paradigm. This scaling limit applies not just to GPUs, but to any architecture where state updates are sequentialized or memory-bound, including most digital neuromorphic chips. Furthermore, recent proposals to enhance expressivity by employing Stuart-Landau oscillators \citep{zhang2025slgnn}---which require simulating amplitude dynamics in addition to phase---will only increase the computational load per node, making the thermodynamic bottleneck identified here even more acute.

This does not diminish HoloGraph's contribution. Rather, it clarifies the path forward: \emph{oscillatory neural networks require dedicated hardware to scale}.

\subsection{Neuromorphic Implementation Pathways}

Several substrate technologies could implement Coherence Gate hardware:

\textbf{CMOS oscillator arrays:} LC tank oscillators or ring oscillators with resistive/capacitive coupling. Quality factors $Q \sim 10^2$--$10^3$ achievable in standard processes.

\textbf{Photonic implementations:} Optical ring resonators with evanescent coupling. Quality factors $Q \sim 10^5$--$10^6$, enabling massive parallelism via wavelength-division multiplexing.

\textbf{Memristive crossbars:} Phase-change or resistive memory elements providing programmable coupling weights with analog dynamics.

Each pathway offers the key property: coupling occurs through physical component connectivity without algorithmic computation. Mixed-signal oscillatory neural networks such as SKONN \citep{delacour2023skonn} already demonstrate the viability of analog phase-domain computation, while recent work has begun to explore learning rules for physically coupled oscillator networks \citep{rageau2025equilibrium}. Our analysis complements these architectures by quantifying the scaling advantage of physically instantiated coupling over digital simulation.

\subsection{Distinction from Physical Reservoir Computing}

Our approach is distinct from physical reservoir computing, which typically leverages the high-dimensional, non-linear dynamics of a physical substrate (e.g., nanowire networks or memristive fabrics) to project inputs into a higher-dimensional feature space. While reservoirs exploit physical complexity for \emph{separability}, the Coherence Gate exploits physical synchronization for \emph{reducibility}. We use the substrate's tendency toward order (collapse onto coherent attractors) to bind features, rather than its tendency toward chaos (expansion into high-dimensional trajectories) to separate them. This collapse--not expansion--is what enables the $O(K)$ coherence monitoring to suffice.

\subsection{Relation to Thermodynamic Computing}

Our results sit within a broader shift toward physics-based computing. Recent thermodynamic computing proposals, including magnetic thermodynamic cores and commercial thermodynamic sampling units (TSUs) such as those developed by Extropic, exploit stochastic device physics to perform probabilistic sampling and inference far more efficiently than digital simulation. These approaches primarily leverage thermal noise as a computational resource for energy-based models. By contrast, the Coherence Gate specifically targets \textit{relational structure} through oscillatory synchronization. Rather than sampling from a distribution, the Coherence Gate uses physical coupling to perform deterministic (or metastable) binding of features, replacing $O(N^2)$ algorithmic updates with substrate-driven synchronization. This suggests a hybrid future where TSUs generate candidates via sampling, and Coherence Gates stabilize and bind those candidates into discrete symbolic codes.

\subsection{Biological Relevance}

The Coherence Gate architecture mirrors observations in biological neural systems. Gamma-frequency oscillations (30--100 Hz) during perceptual binding \citep{buzsaki2004neuronal, fries2015rhythms} exhibit coherence-threshold dynamics: neural populations explore high-dimensional configuration space, then collapse to coherent states that are communicated downstream. The brain does not ``simulate'' oscillator dynamics---it \emph{is} an oscillator network.

\section{Conclusion}

We have established fundamental scaling limits for software-based oscillatory neural networks. Digital simulation of continuous coupling dynamics incurs $O(N^2)$ computational cost per timestep, rendering large-scale implementations thermodynamically intractable. Physical instantiation of oscillator coupling eliminates this barrier, achieving $10^2$--$10^3\times$ power reduction at current scales with advantages growing as $O(N^2/K)$.

The Coherence Gate provides a concrete architectural primitive for realizing this advantage: physical coupling, sparse coherence monitoring, threshold-triggered registration, and rate budgeting combine to enable scalable oscillatory computing.

The success of HoloGraph and related approaches validates oscillatory synchronization as a powerful computational mechanism. Our results clarify that realizing this potential at scale requires moving beyond simulation to dedicated neuromorphic hardware that physically instantiates the coupling term.

\section*{Acknowledgments}

None.

\section*{Data Availability}

Simulation code and data are available at: \url{https://github.com/todd866/coherence-gate}

\bibliographystyle{unsrtnat}
\bibliography{references}

\end{document}
